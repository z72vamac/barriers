\documentclass[a4paper]{elsarticle}
\usepackage[utf8]{inputenc}
\usepackage[T1]{fontenc}
%\usepackage[spanish]{babel}
\usepackage{amsmath}
\usepackage{amsfonts}
\usepackage{amssymb}
\usepackage{graphicx}
\usepackage{mathtools,amssymb}
\usepackage{subfigure}
\usepackage{optidef}
\usepackage{xcolor}
\usepackage{amsthm}
\usepackage{comment}
\usepackage{tikz}
\usepackage{pgfplots}
\usepackage{mathrsfs}
\usepackage{float}
\usepackage[linesnumbered,ruled,vlined]{algorithm2e}
%\usepackage[noend]{algpseudocode}
%\usepackage{ulem}
\usepackage[margin=1in]{geometry}
\usepackage{enumitem}


\usepackage{threeparttable}

\usepackage[utf8]{inputenc}

\title{The Traveling Postman Problem with Linear Barriers}
\author{carlosvalverdemartin }
\date{February 2021}

\DeclareMathOperator*{\argmax}{arg\,max}
\DeclareMathOperator*{\argmin}{arg\,min}


\newcommand{\SPP}{{\sf{SPPN-P} \xspace}}
\newcommand{\TSP}{{\sf{TSPN-P} \xspace}}
\newcommand{\B}{{\mathcal B}}
\newcommand{\VB}{{V^{}_{\mathcal B}}}
\newcommand{\EB}{{E^{}_{\mathcal B}}}
\newcommand{\VS}{{V^{}_{S}}}
\newcommand{\ES}{{E^{}_{S}}}
\newcommand{\VT}{{V^{}_{T}}}
\newcommand{\ET}{{E^{}_{T}}}
\newcommand{\VN}{{V^{}_{\mathcal N}}}
\newcommand{\EN}{{E^{}_{\mathcal N}}}

\newtheorem{remark}{Remark}
\newtheorem{notation}{Notation}

\newtheorem{prop}{Proposition}



\begin{document}
\section{Description of the Problem}\label{section:description}
In this section, it is described the two problems that are considered in this paper: the Shortest Path Problem with Neighborhoods in a Poligonal Domain \SPP and the Traveling Salesman Problem with Neighborhoods in a Poligonal Domain \TSP. 

%In \SPP, we have a source neighborhood $N_S\subset\mathbb R^2$ and a target neighborhood $N_T\subset\mathbb R^2$, that we assume to be convex sets and a set $\mathcal B$ of opened line segments located in general position that plays the role of barriers that the drone cannot cross, i.e., the drone can visit the endpoints of these barriers but it cannot be located in the interior points of the barriers. Note that, it is rational to assume that if there are two barriers that have a portion of them in common, it is only considered the smallest line segment that contains both barriers. The aim of the \SPP is to find the best pair of points $(P_{S}, P_{T})\in N_S\times N_T$ in the source and target neighborhoods that minimize the length of the path that joins both points without crossing any barrier of $\mathcal B$. In this problem, it is implicitly assumed that there is not a rectilinear path to go from $N_S$ to $N_T$, otherwise, the problem becomes straightforward and the solution is the minimum distance between both neighborhoods. 
In \SPP, we have a source neighborhood $N_S\subset\mathbb R^2$ and a target neighborhood $N_T\subset\mathbb R^2$, that we assume to be second order cone representable sets and a set $\mathcal B$ of line segments that plays the role of barriers that the drone cannot cross. In this model we state the following assumptions:

\begin{enumerate}[label=\textbf{A\arabic*},ref=\textbf{A\arabic*}]
\item \label{A1}The line segments of $\mathcal B$ are located in general position, i.e., the endpoints of these segments are not aligned. Although it is possible to model the most general case, it is beyond of the scope of this paper.
\item The line segments of $\mathcal B$ are opened, that is, it is possible that the drone visits the endpoints of the segments, but it cannot be located in the interior points of them. Otherwise, we can take a small quantity to enlarge these segments to make them opened.
\item If there are two barriers that have a common portion of them, it is only considered the smallest line segment that contains both barriers.
\item \label{A4}There is not a rectilinear path to go from $N_S$ to $N_T$. Otherwise, the problem becomes straightforward and the solution is the minimum distance between both neighborhoods.
\end{enumerate}

The aim of the \SPP is to find the best pair of points $(P_{S}, P_{T})\in N_S\times N_T$ in the source and target neighborhoods that minimize the length of the path that joins both points without crossing any barrier of $\mathcal B$ by assuming \ref{A1}-\ref{A4}.


%It is important to remark that it is assumed that these line segments are opened, i.e., the drone can visit $B_b$ or $B_b'$ for the barrier $b$. 
%
%The endpoints of $b\in\mathcal B$ are notated by $B_b$ and $B_b'$.

The \TSP is an extension of the \SPP where a neighborhood set $\mathcal N$ is considered that plays the role of source and target neighborhoods in the \SPP. The aim of the \TSP is to seek the shortest tour that visits each neighborhood $N\in\mathcal N$ exactly once without crossing any barrier $B\in\mathcal B$ by assuming again \ref{A1}-\ref{A4}.

The Figure \ref{fig:initialdata} shows an example of the problem that are being solved. In the left picture, the blue neighborhood represents the source, the green one, representing the \SPP, the target and the red line segments show the barriers that the drone can not cross. In the right picture, an instance of the \TSP is shown, where the neighborhood are balls and the barriers are, again, the red line segments.

\pgfplotsset{compat=1.15}
\usetikzlibrary{arrows}
\definecolor{qqffqq}{rgb}{0,1,0}
\definecolor{qqqqff}{rgb}{0,0,1}
\definecolor{ffqqqq}{rgb}{1,0,0}
\definecolor{bfffqq}{rgb}{0.7490196078431373,1,0}
\definecolor{ffxfqq}{rgb}{1,0.4980392156862745,0}
\definecolor{qqffqq}{rgb}{0,1,0}
\definecolor{qqqqff}{rgb}{0,0,1}
\definecolor{ffqqqq}{rgb}{1,0,0}
\begin{figure}[h!]
\centering
\begin{tikzpicture}[line cap=round,line join=round,>=triangle 45,x=1cm,y=1cm, scale = 0.65]
\begin{axis}[
x=0.1cm,y=0.1cm,
axis lines=middle,
%ymajorgrids=true,
%xmajorgrids=true,
xmin=-5,
xmax=105,
ymin=-5,
ymax=105,
xtick={-30,-20,...,150},
ytick={-20,-10,...,110},]
\clip(-30.53772939574263,-29.679586779798747) rectangle (159.53639158056973,116.68410916363342);
\draw [line width=1pt,color=ffqqqq] (20,80)-- (40,30);
\draw [line width=1pt,color=ffqqqq] (70,95)-- (40,70);
\draw [line width=1pt,color=ffqqqq] (95,60)-- (60,70);
\draw [line width=1pt,color=ffqqqq] (60,50)-- (90,10);
\draw [line width=1pt,color=ffqqqq] (10,70)-- (20,50);
\draw [rotate around={0:(20,10)},line width=1pt,color=qqqqff,fill=qqqqff,fill opacity=0.25] (20,10) ellipse (1cm and 1cm);
\draw [rotate around={0:(90,90)},line width=1pt,color=qqffqq,fill=qqffqq,fill opacity=0.25] (90,90) ellipse (0.5cm and 0.5cm);
\begin{scriptsize}
\draw [color=ffqqqq] (20,80) circle (2.5pt);
\draw [color=ffqqqq] (40,30) circle (2.5pt);
\draw [color=ffqqqq] (70,95) circle (2.5pt);
\draw [color=ffqqqq] (40,70) circle (2.5pt);
\draw [color=ffqqqq] (95,60) circle (2.5pt);
\draw [color=ffqqqq] (60,70) circle (2.5pt);
\draw [color=ffqqqq] (60,50) circle (2.5pt);
\draw [color=ffqqqq] (90,10) circle (2.5pt);
\draw [color=ffqqqq] (10,70) circle (2.5pt);
\draw [color=ffqqqq] (20,50) circle (2.5pt);
\end{scriptsize}
\end{axis}
\end{tikzpicture}
\begin{tikzpicture}[line cap=round,line join=round,>=triangle 45,x=1cm,y=1cm, scale = 0.65]
\begin{axis}[
x=0.1cm,y=0.1cm,
axis lines=middle,
%ymajorgrids=true,
%xmajorgrids=true,
xmin=-5,
xmax=105,
ymin=-5,
ymax=105,
xtick={-30,-20,...,150},
ytick={-20,-10,...,110},]
\clip(-30.53772939574263,-29.679586779798747) rectangle (159.53639158056973,116.68410916363342);
\draw [line width=1pt,color=ffqqqq] (20,80)-- (40,30);
\draw [line width=1pt,color=ffqqqq] (70,95)-- (40,70);
\draw [line width=1pt,color=ffqqqq] (95,60)-- (60,70);
\draw [line width=1pt,color=ffqqqq] (60,50)-- (90,10);
\draw [line width=1pt,color=ffqqqq] (10,70)-- (20,50);
\draw [rotate around={0:(20,10)},line width=1pt,color=qqqqff,fill=qqqqff,fill opacity=0.25] (20,10) ellipse (1cm and 1cm);
\draw [rotate around={0:(90,90)},line width=1pt,color=qqffqq,fill=qqffqq,fill opacity=0.25] (90,90) ellipse (0.5cm and 0.5cm);
\draw [rotate around={0:(35,85)},line width=1pt,color=ffxfqq,fill=ffxfqq,fill opacity=0.25] (35,85) ellipse (0.9cm and 0.9cm);
\draw [rotate around={0:(85,40)},line width=1pt,color=bfffqq,fill=bfffqq,fill opacity=0.25] (85,40) ellipse (1.1cm and 1.1cm);
\begin{scriptsize}
\draw [color=ffqqqq] (20,80) circle (2.5pt);
\draw [color=ffqqqq] (40,30) circle (2.5pt);
\draw [color=ffqqqq] (70,95) circle (2.5pt);
\draw [color=ffqqqq] (40,70) circle (2.5pt);
\draw [color=ffqqqq] (95,60) circle (2.5pt);
\draw [color=ffqqqq] (60,70) circle (2.5pt);
\draw [color=ffqqqq] (60,50) circle (2.5pt);
\draw [color=ffqqqq] (90,10) circle (2.5pt);
\draw [color=ffqqqq] (10,70) circle (2.5pt);
\draw [color=ffqqqq] (20,50) circle (2.5pt);
\end{scriptsize}
\end{axis}
\end{tikzpicture}
\caption{Problem data of the \SPPN and \TSPN}
\label{fig:initialdata}
\end{figure}

\section{MINLP Formulations}

In this section it is introduced a Mixed Integer Non-Linear Programming formulation for the problems described in Section \ref{section:description}. Firstly, we set the constraints that check if two segments have crossing points or not. Then, we present the conic representation of the neighborhoods and the distance. Finally, the formulation for the \SPP is described and the model \TSP is explained as an extension of the previous problem.

\subsection{Checking if two segments intersect}
Let $\overline{PP'}$ and $\overline{QQ'}$ be two line segments that we want to check if they intersect. The following well-known computational geometry result can be used to analyze their relative position:

\newcommand{\segment}[2]{\overline{#1#2}}
\newcommand{\determinant}[3]{\det({#1|\overline{#2#3}})}


\begin{remark}\label{rem:determinants}
Let $\overline{PP'}$ and $\overline{QQ'}$ be two different line segments. Let also denote $\determinant{P}{Q}{Q'}=\det\left(\begin{array}{c|c} \overrightarrow{PQ} & \overrightarrow{PQ'}\end{array}\right)$ the determinant whose arguments are $P$, $Q$ and $Q'$. If
\begin{equation*}
\normalfont{\text{sign}}\left(\determinant{P}{Q}{Q'}\right) = \normalfont{\text{sign}}\left(\determinant{P'}{Q}{Q'}\right)
\quad
\text{or}
\quad
\normalfont{\text{sign}}\left(\determinant{Q}{P}{P'}\right) = \normalfont{\text{sign}}\left(\determinant{Q'}{P}{P'}\right)
,
\end{equation*}
then $\overline{PP'}$ and $\overline{QQ'}$ do not intersect.
\end{remark}

On the following, we are going to model this condition by introducing the necessary binary variables that check the sign of the determinants, the equality of signs and the disjunctive condition.

\newcommand{\LS}[3]{L(#1|\overline{#2#3})}
\newcommand{\US}[3]{U(#1|\overline{#2#3})}
\newcommand{\alphamas}[3]{\alpha(#1|\overline{#2#3})}
\newcommand{\alphamenos}[3]{\alpha^{-}(#1|#2#3)}
%\newcommand{\alphacero}[3]{\alpha^{0\,}(#1|#2#3)}
\newcommand{\alphapunto}[3]{\alpha^{\cdotp}(#1|#2#3)}

To model the sign of a determinant defined by the points $P$, $Q$ and $Q'$, we introduce the binary variable $\alphamas{P}{Q}{Q'}$, that is one if $\determinant{P}{Q}{Q'}$ is positive and zero, otherwise. Note that the case when the determinant is null is not considered , because segments are located in general position.

It is possible to represent the sign condition by including the following constraints:
%\begin{equation}\tag{sign+}\label{eq:sign}
%\LS{P_S^{}}{P_{B'}^1}{P_{B'}^2}\left(1-\alpha{P_S^{}}{P_{B'}^1}{P_{B'}^2}\right)\leq \determinant{P_S^{}}{P_{B'}^1}{P_{B'}^2}\leq \US{P_S^{}}{P_{B'}^1}{P_{B'}^2}\alpha{P_S^{}}{P_{B'}^1}{P_{B'}^2},
%\end{equation}
%\begin{align*}\tag{sign}\label{eq:sign}
%\determinant{P_S^{}}{P_{B'}^1}{P_{B'}^2}&\leq \US{P_S^{}}{P_{B'}^1}{P_{B'}^2}\:\alpha{P_S^{}}{P_{B'}^1}{P_{B'}^2},\\
%\determinant{P_S^{}}{P_{B'}^1}{P_{B'}^2}&\geq \LS{P_S^{}}{P_{B'}^1}{P_{B'}^2}\:\alphamenos{P_S^{}}{P_{B'}^1}{P_{B'}^2},\\
%\alpha{P_S^{}}{P_{B'}^1}{P_{B'}^2}+\alphamenos{P_S^{}}{P_{B'}^1}{P_{B'}^2}& = 1,
%\end{align*}
\begin{equation}\tag{$\alpha$-C}\label{eq:alphaC}
\left[1-\alphamas{P}{Q}{Q'}\right]\LS{P}{Q}{Q'}\leq\determinant{P}{Q}{Q'}\leq \US{P}{Q}{Q'}\:\alphamas{P}{Q}{Q'},
\end{equation}

where $\LS{P}{Q}{Q'}$ and $\US{P}{Q}{Q'}$ are a lower and a upper bound for the determinant, respectively. If the determinant is positive, then $\alphamas{P}{Q}{Q'}$ must be one to make the second inequality feasible. The analagous case happens if the determinant is not positive.

\newcommand{\betamas}[4]{\beta(\overline{#1#2}|\overline{#3#4})}
%\newcommand{\betamenos}[4]{\beta^{-}(#1#2|#3#4)}
%\newcommand{\betacero}[4]{\beta^{0\,}(#1#2|#3#4)}
%\newcommand{\betapunto}[4]{\beta^{\cdotp}(#1#2|#3#4)}

Now, to check if two determinants $\determinant{P}{Q}{Q'}$ and $\determinant{P'}{Q}{Q'}$ have the same sign, it is required to introduce the binary variable $\betamas{P}{P'}{Q}{Q'}$, that is one if $\determinant{P}{Q}{Q'}$ and $\determinant{P'}{Q}{Q'}$ have the same sign, zero otherwise.

\newcommand{\gammaprod}[4]{\gamma(\overline{#1#2}|\overline{#3#4})}

Hence, the $\beta$ variable can be represented by the equivalence constraint of the $\alpha$ variables
\begin{align*}\tag{$\beta$-C}\label{eq:betaC}
\betamas{P}{P'}{Q}{Q'}&=\alphamas{P}{Q}{Q'}\alphamas{P'}{Q}{Q'} + \left[1-\alphamas{P}{Q}{Q'}\right]\left[1-\alphamas{P'}{Q}{Q'}\right],\\
\betamas{P}{P'}{Q}{Q'}&=2\gammaprod{P}{P'}{Q}{Q'} -\alphamas{P}{Q}{Q'}-\alphamas{P'}{Q}{Q'}+1,
%\betamas{P_S^{}}{P_B^i}{P_{B'}^1}{P_{B'}^2}&=1-\left|\alphamas{P_S^{}}{P_{B'}^1}{P_{B'}^2}-\alphamas{P_B^i}{P_{B'}^1}{P_{B'}^2}\right|,
\end{align*}
where $\gammaprod{P}{P'}{Q}{Q'}$ is the auxiliary binary variable that models the product of the $\alpha$ variables that can be linearized by using the following constraints:
\begin{align*}\tag{$\gamma$-C}\label{eq:gammaC}
\gammaprod{P}{P'}{Q}{Q'} & \leq \alphamas{P}{Q}{Q'},\\
\gammaprod{P}{P'}{Q}{Q'} & \leq \alphamas{P'}{Q}{Q'},\\
\gammaprod{P}{P'}{Q}{Q'} & \geq \alphamas{P}{Q}{Q'} + \alphamas{P'}{Q}{Q'} - 1.
\end{align*}
 

\newcommand{\deltacheck}[4]{\delta(\overline{#1#2}|\overline{#3#4})}

Finally, we need to check if there exists any coincidence in the sign of the determinants, so we define the binary variable $\deltacheck{P}{P'}{Q}{Q'}$ that is one if $\segment{P}{P'}$ and $\segment{Q}{Q'}$ do not intersect and zero, otherwise. This condition can be modelled by using these disjunctive constraints:
\begin{equation*}\tag{$\delta$-C}\label{eq:deltaC}
\frac{1}{2}\left[\betamas{P}{P'}{Q}{Q'}+\betamas{Q}{Q'}{P}{P'}\right]\leq\deltacheck{P}{P'}{Q}{Q'}\leq 2\left[\betamas{P}{P'}{Q}{Q'}+\betamas{Q}{Q'}{P}{P'}\right].
\end{equation*}
If there exists a sign coincidence, then $\deltacheck{P}{P'}{Q}{Q'}$ is one to satisfy the first constraint, and the second one is always fullfilled. However, if the sign of the determinants is not the same, then the second constraint is active and $\deltacheck{P}{P'}{Q}{Q'}$ is null.

\subsection{Conic constraints in the models}
In both problems, there exist two second-order cone constraints that model the distance between a pair of points $P$ and $Q$ and the representation of the neighborhoods where the points can be selected.

\newcommand{\dvar}[2]{d(#1#2)}

For the former case, we introduce the non-negative continuous variable $\dvar{PQ}$ that represents the distance between $P$ and $Q$:


\begin{equation*}\tag{d-C}\label{eq:dC}
\|P - Q\|\leq \dvar{P}{Q},\quad\forall (P,Q)\in E.
\end{equation*}

For the latter case, since we are assuming that the neighborhoods are second order cone (SOC) representable, they can be expressed by means of the constraints:

\begin{equation*}\tag{N-C}\label{eq:nC}
 P^{}_N\in N \Longleftrightarrow
  \|A_N^i P_N^{} + b_N^i\| \leq (c_N^i)^T P_N^{} + d_N^i,\quad i=1,\ldots,nc_N, \\
\end{equation*}
%\begin{equation}\label{C-C}\tag{$\mathcal{C}$-C}
%    \|B_ix + b_i\|\leq c_i^Tx + d_i,\quad i=1,\ldots,N,
%\end{equation}
where $A_N^i, b_N^i, c_N^i$ and $d_N^i$ are parameters of the constraint $i$ and $nc_N$ denotes the number of constraints that appear in the block associated to the neighborhood $N$.

It is remarkable that these inequalities can model the special case of linear constraints (for $A_N^{i}, b_N^i\equiv 0$), ellipsoids and hyperbolic constraints (see \cite{Lobo1998} for more information).

\textcolor{red}{Puede ser muy interesante el caso 3D pero quizás en otro trabajo, no?}

\subsection{A formulation for the \SPP}
The main idea of the \SPP is to solve a shortest path problem in the undirected graph induced by the endpoints of the barriers and the neighborhoods. Here, it is necessary to define the following sets:
\begin{itemize}
\item $\VS=\{P_S\}$: set composed by the point selected in the source neighborhood $N_S$.
\item $\VB=\{P^1_B, P^2_B:B=\overline{P^1_B P^2_B}\in \mathcal B\}$: set of vertices that form the barriers of the problem.
\item $\VT=\{P^{}_T\}$: set composed by the point selected in the target neighborhood $N_T$.
\item $\ES=\{(P_S, P^i_{B}):P^i_B\in V_\B\text{ and } \overline{P_SP^i_B}\cap B''=\emptyset,\forall B''\in\B,\:i=1,2\}$: set of edges formed by the line segments that join the point selected in the source neighborhood and every endpoint in the barriers and do not cross any barrier in $\B$.
\item $\EB=\{(P^{i}_B, P^{j}_{B'}):P^i_B, P^j_{B'}\in \VB \text{ and } \overline{P^i_B P^j_{B'}}\cap B''=\emptyset,\:\forall B''\in\mathcal B,\:i, j=1,2\}$: set of edges formed by the line segments that join two vertices of $V_{\mathcal B}$ and do not cross any barrier in $\B$.
\item $\ET=\{(P^{}_T, P^i_{B}):P^i_B\in V_\B\text{ and } \overline{P^{}_TP^i_B}\cap B''=\emptyset,\forall B''\in\B,\:i=1,2\}$: set of edges formed by the line segments that join the point selected in the target neighborhood and every endpoint in the barriers and do not cross any barrier in $\B$.
\textcolor{red}{Definimos tambien la arista que pueda unir a los dos entornos o asumimos que no hay un camino que los una? Depende del caso que consideremos el problema es convexo o no.}
\end{itemize} 

At this point, we can define the graph $G= (V, E)$ induced by the barriers and the neighborhoods, where $V=\VS\cup \VB\cup\VT$ and $E=\ES\cup\EB \cup\ET$. It is interesting to note that this graph can be split into two parts: a fixed graph $G_\B=(\VB,\EB)$ whose edges can be computed by using the Remark \ref{rem:determinants} and the sets $\VS$, $\ES$, $\VT$ and $\ET$ that depend on where the points $P_S$ and $P^{}_T$ are located as shown in Figure.  The figures show how the graph $G$ is generated. The blue dashed line segments represent the edges of $\ES$, the green ones, the edges of $\ET$ and the red dashed lines, the edges of $\EB$. A special case that can be remarked occurs when the neighborhoods are points. In that case, the induced graph is completely fixed and it is only necessary to find which edges are included by keeping in mind that there can not have crossings. This idea is exploited in the Subsection \ref{section:reformulation}.

%\textcolor{red}{Me estoy dando cuenta que nos pueden decir que para qué tenemos en cuenta todo el conjunto si realmente se pueden coger los puntos de la frontera mas cerca a cada uno de los puntos por donde puede salir de las barreras. Es cierto que solo ocurre con el SPP. En el caso del TSP esto no es cierto.}

\pgfplotsset{compat=1.15}
\usetikzlibrary{arrows}
\begin{figure}[h!]
\centering
\definecolor{qqffqq}{rgb}{0,1,0}
\definecolor{qqqqff}{rgb}{0,0,1}
\definecolor{ffqqqq}{rgb}{1,0,0}
\begin{tikzpicture}[line cap=round,line join=round,>=triangle 45,x=1cm,y=1cm, scale=0.65]
\begin{axis}[
x=0.1cm,y=0.1cm,
axis lines=middle,
xmin=-5,
xmax=105,
ymin=-5,
ymax=105,
xtick={-30,-20,...,160},
ytick={-30,-20,...,100},]
\clip(-30.537729395742623,-31.914778968980123) rectangle (160.8609499148995,105.01143884235292);
\draw [line width=1pt,color=ffqqqq] (20,80)-- (40,30);
\draw [line width=1pt,color=ffqqqq] (70,95)-- (40,70);
\draw [line width=1pt,color=ffqqqq] (95,60)-- (60,70);
\draw [line width=1pt,color=ffqqqq] (60,50)-- (90,10);
\draw [line width=1pt,color=ffqqqq] (10,70)-- (20,50);
\draw [line width=1pt,dashed,color=ffqqqq] (10,70)-- (20,80);
\draw [line width=1pt,dashed,color=ffqqqq] (20,50)-- (40,30);
\draw [line width=1pt,dashed,color=ffqqqq] (90,10)-- (40,30);
\draw [line width=1pt,dashed,color=ffqqqq] (20,80)-- (40,70);
\draw [line width=1pt,dashed,color=ffqqqq] (20,80)-- (70,95);
\draw [line width=1pt,dashed,color=ffqqqq] (40,70)-- (40,30);
\draw [line width=1pt,dashed,color=ffqqqq] (40,70)-- (60,70);
\draw [line width=1pt,dashed,color=ffqqqq] (70,95)-- (95,60);
\draw [line width=1pt,dashed,color=ffqqqq] (95,60)-- (90,10);
\draw [line width=1pt,dashed,color=ffqqqq] (60,50)-- (40,30);
\draw [line width=1pt,dashed,color=ffqqqq] (40,30)-- (60,70);
\draw [line width=1pt,dashed,color=ffqqqq] (60,70)-- (70,95);
\draw [line width=1pt,dashed,color=ffqqqq] (20,50)-- (20,80);
\draw [rotate around={0:(20,10)},line width=1pt,color=qqqqff,fill=qqqqff,fill opacity=0.25] (20,10) ellipse (1cm and 1cm);
\draw [rotate around={0:(90,90)},line width=1pt,color=qqffqq,fill=qqffqq,fill opacity=0.25] (90,90) ellipse (0.5cm and 0.5cm);
\draw [line width=1pt,dashed,color=ffqqqq] (10,70)-- (40,30);
\draw [line width=1pt,dashed,color=ffqqqq] (60,50)-- (60,70);
\draw [line width=1pt,dashed,color=ffqqqq] (60,50)-- (95,60);
\draw [line width=1pt,dashed,color=ffqqqq] (90,10)-- (60,70);
\draw [line width=1pt,dashed,color=qqqqff] (12.786085173820345,16.92527493177169)-- (10,70);
\draw [line width=1pt,dashed,color=qqqqff] (12.786085173820345,16.92527493177169)-- (20,50);
\draw [line width=1pt,dashed,color=qqqqff] (12.786085173820345,16.92527493177169)-- (40,30);
\draw [line width=1pt,dashed,color=qqqqff] (12.786085173820345,16.92527493177169)-- (90,10);
\draw [line width=1pt,dashed,color=qqffqq] (89.83150923646912,88.53724455912725)-- (70,95);
\draw [line width=1pt,dashed,color=qqffqq] (89.83150923646912,88.53724455912725)-- (40,70);
\draw [line width=1pt,dashed,color=qqffqq] (89.83150923646912,88.53724455912725)-- (60,70);
\draw [line width=1pt,dashed,color=qqffqq] (89.83150923646912,88.53724455912725)-- (95,60);
\draw [color=qqqqff](3.9007872968296593,8.815389811658244) node[anchor=north west] {$\mathbf{N_S}$};
\draw [color=qqffqq](95.46088215737039,101.20333363115502) node[anchor=north west] {$\mathbf{N_T}$};
\draw (87.18239256780974,95.5) node[anchor=north west] {$\mathbf{P_T}$};
\draw (5.722055006533001,19.742996069878295) node[anchor=north west] {$\mathbf{P_S}$};
\draw [line width=1pt,dashed,color=ffqqqq] (40,70)-- (60,50);
\draw [line width=1pt,dashed,color=ffqqqq] (60,50)-- (20,80);
\draw [line width=1pt,dashed,color=ffqqqq] (20,80)-- (90,10);
\draw [color=ffqqqq](68.5,53.8) node[anchor=north west] {$\mathbf{G_{\mathcal B}=(V_{\mathcal B}, E_{\mathcal B})}$};
\draw [color=qqffqq](68,90.11015758114377) node[anchor=north west] {$\mathbf{E_T}$};
\draw [color=qqqqff](21.1200456431158,33.9819981639226) node[anchor=north west] {$\mathbf{E_S}$};
\draw [line width=1pt,dashed,color=ffqqqq] (40,70)-- (95,60);
\draw [line width=1pt,dashed,color=ffqqqq] (40,70)-- (90,10);
\draw [line width=1pt,dashed,color=ffqqqq] (40,30)-- (70,95);
\begin{scriptsize}
\draw [color=ffqqqq] (20,80) circle (2.5pt);
\draw [color=ffqqqq] (40,30) circle (2.5pt);
\draw [color=ffqqqq] (70,95) circle (2.5pt);
\draw [color=ffqqqq] (40,70) circle (2.5pt);
\draw [color=ffqqqq] (95,60) circle (2.5pt);
\draw [color=ffqqqq] (60,70) circle (2.5pt);
\draw [color=ffqqqq] (60,50) circle (2.5pt);
\draw [color=ffqqqq] (90,10) circle (2.5pt);
\draw [color=ffqqqq] (10,70) circle (2.5pt);
\draw [color=ffqqqq] (20,50) circle (2.5pt);
\draw [fill=qqqqff] (12.786085173820345,16.92527493177169) circle (2.5pt);
\draw [fill=qqffqq] (89.83150923646912,88.53724455912725) circle (2.5pt);
\end{scriptsize}
\end{axis}
\end{tikzpicture}
\begin{tikzpicture}[line cap=round,line join=round,>=triangle 45,x=1cm,y=1cm, scale=0.65]
\begin{axis}[
x=0.1cm,y=0.1cm,
axis lines=middle,
xmin=-5,
xmax=105,
ymin=-5,
ymax=105,
xtick={-30,-20,...,160},
ytick={-30,-20,...,95},]
\clip(-30.537729395742623,-36.63351803502969) rectangle (160.8609499148995,109.73017790840248);
\draw [line width=1pt,color=ffqqqq] (20,80)-- (40,30);
\draw [line width=1pt,color=ffqqqq] (70,95)-- (40,70);
\draw [line width=1pt,color=ffqqqq] (95,60)-- (60,70);
\draw [line width=1pt,color=ffqqqq] (60,50)-- (90,10);
\draw [line width=1pt,color=ffqqqq] (10,70)-- (20,50);
\draw [line width=1pt,dashed,color=ffqqqq] (10,70)-- (20,80);
\draw [line width=1pt,dashed,color=ffqqqq] (20,50)-- (40,30);
\draw [line width=1pt,dashed,color=ffqqqq] (90,10)-- (40,30);
\draw [line width=1pt,dashed,color=ffqqqq] (20,80)-- (40,70);
\draw [line width=1pt,dashed,color=ffqqqq] (20,80)-- (70,95);
\draw [line width=1pt,dashed,color=ffqqqq] (40,70)-- (40,30);
\draw [line width=1pt,dashed,color=ffqqqq] (40,70)-- (60,70);
\draw [line width=1pt,dashed,color=ffqqqq] (70,95)-- (95,60);
\draw [line width=1pt,dashed,color=ffqqqq] (95,60)-- (90,10);
\draw [line width=1pt,dashed,color=ffqqqq] (60,50)-- (40,30);
\draw [line width=1pt,dashed,color=ffqqqq] (40,30)-- (60,70);
\draw [line width=1pt,dashed,color=ffqqqq] (60,70)-- (70,95);
\draw [line width=1pt,dashed,color=ffqqqq] (20,50)-- (20,80);
\draw [rotate around={0:(20,10)},line width=1pt,color=qqqqff,fill=qqqqff,fill opacity=0.25] (20,10) ellipse (1cm and 1cm);
\draw [rotate around={0:(90,90)},line width=1pt,color=qqffqq,fill=qqffqq,fill opacity=0.25] (90,90) ellipse (0.5cm and 0.5cm);
\draw [line width=1pt,dashed,color=ffqqqq] (10,70)-- (40,30);
\draw [line width=1pt,dashed,color=ffqqqq] (60,50)-- (60,70);
\draw [line width=1pt,dashed,color=ffqqqq] (60,50)-- (95,60);
\draw [line width=1pt,dashed,color=ffqqqq] (90,10)-- (60,70);
\draw [line width=1pt,dashed,color=qqqqff] (27.080558147599465,12.871849710542959)-- (10,70);
\draw [line width=1pt,dashed,color=qqqqff] (27.080558147599465,12.871849710542959)-- (20,50);
\draw [line width=1pt,dashed,color=qqqqff] (27.080558147599465,12.871849710542959)-- (40,30);
\draw [line width=1pt,dashed,color=qqqqff] (27.080558147599465,12.871849710542959)-- (90,10);
\draw [line width=1pt,dashed,color=qqffqq] (89.83150923646912,88.53724455912725)-- (70,95);
\draw [line width=1pt,dashed,color=qqffqq] (89.83150923646912,88.53724455912725)-- (40,70);
\draw [line width=1pt,dashed,color=qqffqq] (89.83150923646912,88.53724455912725)-- (60,70);
\draw [line width=1pt,dashed,color=qqffqq] (89.83150923646912,88.53724455912725)-- (95,60);
\draw [color=qqqqff](3.9007872968296593,8.898174707553851) node[anchor=north west] {$\mathbf{N_S}$};
\draw [color=qqffqq](95.46088215737039,101.12054873525942) node[anchor=north west] {$\mathbf{N_T}$};
\draw (87.18239256780974,95.5) node[anchor=north west] {$\mathbf{P_T}$};
\draw (20.457766475950947,15.024257003828726) node[anchor=north west] {$\mathbf{P_S}$};
\draw [line width=1pt,dashed,color=ffqqqq] (40,70)-- (60,50);
\draw [line width=1pt,dashed,color=ffqqqq] (60,50)-- (20,80);
\draw [line width=1pt,dashed,color=ffqqqq] (20,80)-- (90,10);
\draw [line width=1pt,dashed,color=qqqqff] (27.080558147599465,12.871849710542959)-- (20,80);
\draw [line width=1pt,dashed,color=qqqqff] (27.080558147599465,12.871849710542959)-- (60,50);
\draw [color=ffqqqq](68.5,53.8) node[anchor=north west] {$\mathbf{G_{\mathcal B}=(V_{\mathcal B}, E_{\mathcal B})}$};
\draw [color=qqffqq](68,90.19294247703937) node[anchor=north west] {$\mathbf{E_T}$};
\draw [color=qqqqff](28.736256065511594,33.56807368444457) node[anchor=north west] {$\mathbf{E_S}$};
\draw [line width=1pt,dashed,color=ffqqqq] (40,70)-- (95,60);
\draw [line width=1pt,dashed,color=ffqqqq] (40,70)-- (90,10);
\draw [line width=1pt,dashed,color=ffqqqq] (70,95)-- (40,30);
\begin{scriptsize}
\draw [color=ffqqqq] (20,80) circle (2.5pt);
\draw [color=ffqqqq] (40,30) circle (2.5pt);
\draw [color=ffqqqq] (70,95) circle (2.5pt);
\draw [color=ffqqqq] (40,70) circle (2.5pt);
\draw [color=ffqqqq] (95,60) circle (2.5pt);
\draw [color=ffqqqq] (60,70) circle (2.5pt);
\draw [color=ffqqqq] (60,50) circle (2.5pt);
\draw [color=ffqqqq] (90,10) circle (2.5pt);
\draw [color=ffqqqq] (10,70) circle (2.5pt);
\draw [color=ffqqqq] (20,50) circle (2.5pt);
\draw [fill=qqqqff] (27.080558147599465,12.871849710542959) circle (2.5pt);
\draw [fill=qqffqq] (89.83150923646912,88.53724455912725) circle (2.5pt);
\end{scriptsize}
\end{axis}
\end{tikzpicture}
\caption{The construction of the graph $G=(N,V)$}
\label{fig:graph}
\end{figure}
%\pgfplotsset{compat=1.15}
\usetikzlibrary{arrows}
\definecolor{qqffqq}{rgb}{0,1,0}
\definecolor{qqqqff}{rgb}{0,0,1}
\definecolor{ffqqqq}{rgb}{1,0,0}
\begin{figure}[h!]
\centering
\begin{tikzpicture}[line cap=round,line join=round,>=triangle 45,x=1cm,y=1cm, scale = 0.5]
\begin{axis}[
x=0.1cm,y=0.1cm,
axis lines=middle,
xmin=-5,
xmax=105,
ymin=-5,
ymax=105,
xtick={-30,-20,...,160},
ytick={-30,-20,...,100},]
\clip(-30.537729395742623,-36.63351803502969) rectangle (160.8609499148995,109.73017790840248);
\draw [line width=1pt,color=ffqqqq] (20,80)-- (40,30);
\draw [line width=1pt,color=ffqqqq] (70,100)-- (40,70);
\draw [line width=1pt,color=ffqqqq] (100,60)-- (60,70);
\draw [line width=1pt,color=ffqqqq] (60,50)-- (90,10);
\draw [line width=1pt,color=ffqqqq] (10,70)-- (20,50);
\draw [line width=1pt,dashed,color=ffqqqq] (10,70)-- (20,80);
\draw [line width=1pt,dashed,color=ffqqqq] (20,50)-- (40,30);
\draw [line width=1pt,dashed,color=ffqqqq] (90,10)-- (40,30);
\draw [line width=1pt,dashed,color=ffqqqq] (20,80)-- (40,70);
\draw [line width=1pt,dashed,color=ffqqqq] (20,80)-- (70,100);
\draw [line width=1pt,dashed,color=ffqqqq] (40,70)-- (40,30);
\draw [line width=1pt,dashed,color=ffqqqq] (40,70)-- (60,70);
\draw [line width=1pt,dashed,color=ffqqqq] (70,100)-- (100,60);
\draw [line width=1pt,dashed,color=ffqqqq] (100,60)-- (90,10);
\draw [line width=1pt,dashed,color=ffqqqq] (60,50)-- (40,30);
\draw [line width=1pt,dashed,color=ffqqqq] (40,30)-- (60,70);
\draw [line width=1pt,dashed,color=ffqqqq] (60,70)-- (70,100);
\draw [line width=1pt,dashed,color=ffqqqq] (20,50)-- (20,80);
\draw [rotate around={0:(20,10)},line width=1pt,color=qqqqff,fill=qqqqff,fill opacity=0.25] (20,10) ellipse (1cm and 1cm);
\draw [rotate around={0:(90,90)},line width=1pt,color=qqffqq,fill=qqffqq,fill opacity=0.25] (90,90) ellipse (0.5cm and 0.5cm);
\draw [line width=1pt,dashed,color=ffqqqq] (10,70)-- (40,30);
\draw [line width=1pt,dashed,color=ffqqqq] (60,50)-- (60,70);
\draw [line width=1pt,dashed,color=ffqqqq] (60,50)-- (100,60);
\draw [line width=1pt,dashed,color=ffqqqq] (90,10)-- (60,70);
\draw [line width=1pt,dashed,color=qqqqff] (27.080558147599465,12.871849710542959)-- (10,70);
\draw [line width=1pt,dashed,color=qqqqff] (27.080558147599465,12.871849710542959)-- (20,50);
\draw [line width=1pt,dashed,color=qqqqff] (27.080558147599465,12.871849710542959)-- (40,30);
\draw [line width=1pt,dashed,color=qqqqff] (27.080558147599465,12.871849710542959)-- (90,10);
\draw [line width=1pt,dashed,color=qqffqq] (89.83150923646912,88.53724455912725)-- (70,100);
\draw [line width=1pt,dashed,color=qqffqq] (89.83150923646912,88.53724455912725)-- (40,70);
\draw [line width=1pt,dashed,color=qqffqq] (89.83150923646912,88.53724455912725)-- (60,70);
\draw [line width=1pt,dashed,color=qqffqq] (89.83150923646912,88.53724455912725)-- (100,60);
\draw [color=qqqqff](3.9007872968296593,8.898174707553851) node[anchor=north west] {$\mathbf{N_S}$};
\draw [color=qqffqq](95.46088215737039,101.12054873525942) node[anchor=north west] {$\mathbf{N_T}$};
\draw (87.18239256780974,96.48459456510545) node[anchor=north west] {$\mathbf{P_T}$};
\draw (20.457766475950947,15.024257003828726) node[anchor=north west] {$\mathbf{P_S}$};
\draw [line width=1pt,dashed,color=ffqqqq] (40,70)-- (60,50);
\draw [line width=1pt,dashed,color=ffqqqq] (60,50)-- (20,80);
\draw [line width=1pt,dashed,color=ffqqqq] (20,80)-- (90,10);
\draw [line width=1pt,dashed,color=qqqqff] (27.080558147599465,12.871849710542959)-- (20,80);
\draw [line width=1pt,dashed,color=qqqqff] (27.080558147599465,12.871849710542959)-- (60,50);
\draw [color=ffqqqq](72.11554151480937,53.76758828297254) node[anchor=north west] {$\mathbf{G_{\mathcal B}=(V_{\mathcal B}, E_{\mathcal B})}$};
\draw [color=qqffqq](69.13528526256754,90.19294247703937) node[anchor=north west] {$\mathbf{E_T}$};
\draw [color=qqqqff](28.736256065511594,33.56807368444457) node[anchor=north west] {$\mathbf{E_S}$};
\draw [line width=1pt,dashed,color=ffqqqq] (40,70)-- (100,60);
\draw [line width=1pt,dashed,color=ffqqqq] (40,70)-- (90,10);
\draw [line width=1pt,dashed,color=ffqqqq] (70,100)-- (40,30);
\begin{scriptsize}
\draw [color=ffqqqq] (20,80) circle (2.5pt);
\draw [color=ffqqqq] (40,30) circle (2.5pt);
\draw [color=ffqqqq] (70,100) circle (2.5pt);
\draw [color=ffqqqq] (40,70) circle (2.5pt);
\draw [color=ffqqqq] (100,60) circle (2.5pt);
\draw [color=ffqqqq] (60,70) circle (2.5pt);
\draw [color=ffqqqq] (60,50) circle (2.5pt);
\draw [color=ffqqqq] (90,10) circle (2.5pt);
\draw [color=ffqqqq] (10,70) circle (2.5pt);
\draw [color=ffqqqq] (20,50) circle (2.5pt);
\draw [fill=qqqqff] (27.080558147599465,12.871849710542959) circle (2.5pt);
\draw [fill=qqffqq] (89.83150923646912,88.53724455912725) circle (2.5pt);
\end{scriptsize}
\end{axis}
\end{tikzpicture}
\end{figure}


%\newcommand{\sign}{{\text{sign}}}


%\newcommand{\determinant}[3]{\det({#1#2#3})}



Since $\ES$ and $\ET$ are not fixed, the determinants in Remark \ref{rem:determinants} also depend on the location of $P_S$ and $P^{}_T$.  Hence, it is essential to model the previous constraint by using binary variables. We only focus on the case of $E_S$ but the same rationale is used for $E_T$.

Let $B\in\B$ be a barrier and $P_B^i$ an endpoint of $B$. Hence, the edge $(P^{}_S, P_B^i)$ belongs to $\ES$ if
$$\overline{P^{}_SP^i_B}\cap B''=\emptyset,\quad \forall B''\in\B,$$
or,
%Firstly, we will expose the constraints for two line segments $\segment{A}{B}$ and $\segment{C}{D}$ in general to model the previous remark. Afterwards, we will focus on the case for edges of $\ES$. 
by the preceding subsection if

%Let $B\in\B$ a barrier and $P_B^i$ and endpoint of $B$. Hence, the edge $(P_S, P_B^i)$ belongs to $\ES$ if $\overline{P^{}_SP^i_B}\cap B'=\emptyset$, for all $B'\in\B$
%or equivalently, by the Remark \ref{rem:determinants}:
%\begin{equation*}
%\normalfont{\text{sign}}\left(\determinant{P_S^{}}{P_{B'}^1}{P_{B'}^2}\right) = \normalfont{\text{sign}}\left(\determinant{P_B^i}{P_{B'}^1}{P_{B'}^2}\right)
%\quad
%\text{or}
%\quad
%\normalfont{\text{sign}}\left(\determinant{P^{}_S}{P_{B'}^1}{P_{B'}^2}\right) = \normalfont{\text{sign}}\left(\determinant{P_B^i}{P_{B'}^1}{P_{B'}^2}\right),
%\quad\forall B'\in\B.
%\end{equation*}
%$$=\emptyset,\quad\forall B''\in\B,$$

%However, if those segments intersect, we need to ensure that $\deltacheck{P_S^{}}{P_B^i}{P_{B'}^1}{P_{B'}^2}$ is zero. It can be done by inserting the term $\epsilon \deltacheck{P_S^{}}{P_B^i}{P_{B'}^1}{P_{B'}^2}$ in the objective function, where $\epsilon>0$ is a small value.
\newcommand{\varepsilonvar}[2]{\varepsilon(#1#2)}
% At this point, we will focus on the description of the edges of $\ES$.
$$\deltacheck{P^{}_S}{P_B^i}{P^1_{B''}}{P^2_{B''}}=1,\quad\forall B''\in\B.$$
Hence, if we denote by $\varepsilonvar{P^{}_S}{P_B^i}$ the binary variable that is one when $(P^{}_S,P_B^i)\in\ES$, this variable can be represented by means of the following inequalities:
\begin{equation*}\tag{$\varepsilon$-C}\label{eq:varepsilonC}
\left[\sum_{B''\in\mathcal B}\deltacheck{P^{}_S}{P_B^i}{P^1_{B''}}{P^2_{B''}}-|\mathcal B|\right] + 1\leq \varepsilonvar{P^{}_S}{P_B^i}\leq \frac{1}{|\B|}\sum_{B''\in\mathcal B}\deltacheck{P^{}_S}{P_B^i}{P^1_{B''}}{P^2_{B''}}.
\end{equation*}
If there is a barrier $B'\in\B$ that intersects the segment $\overline{P^{}_SP_B^i}$, then $\deltacheck{P^{}_S}{P_B^i}{P^1_{B''}}{P^2_{B''}}$ is zero and the second inequality enforces $\varepsilonvar{P^{}_S}{P_B^i}$ to be zero because the right side is fractional and the first inequality is non-positive. Nonetheless, if there is not any barrier that intersects the segment, then $\varepsilonvar{P^{}_S}{P_B^i}$ is equals to one, because the left side of the first inequality is one and the right side of the second inequality too.

\newcommand{\yvar}[2]{y(#1#2)}

Now, we can define the path that the drone can follow by taking into account the edges of the induced graph. Let $\yvar{PQ}$ be the binary variable that is one if the drone goes from $P$ to $Q$. Then, the inequalities
\begin{equation*}\tag{y-C}\label{eq:yC}
\yvar{P^{}_S}{P_B^i}\leq \varepsilonvar{P^{}_S}{P_B^i},\quad\forall P_B^i\in \VB.
\end{equation*}
assure that the drone can go from $P_S^{}\in \VS$ to a point of a barrier only if it does not cross any barrier. 


At this time, we have all the necessary elements to give a MINLP formulation for the \SPP as follows:
%\begin{align*}\tag{H-SPP-S}
%&\text{min} &\sum_{(P,P_B)\in E}\dvar{P}{P_B}\yvar{P}{P_B}&\\
%&\text{s.a.}  \quad &\sum_{\{P_B:(P, P_B)\in E\}}\yvar{P}{P_B}-\sum_{\{P_B:(P_B, P)\in E\}}\yvar{P_B}{P} &= 
%\left\{
%\begin{array}{rl} 
%1, & \text{if } P\in\VS, \\
%0, & \text{if } P\in\VB, \\
%-1, & \text{if } P\in\VT.
%\end{array}.
%\right.\\\\
%& & P_S\in N_S,\, P_T\in N_T\\
%& & \eqref{eq:alphaC},\eqref{eq:betaC},\eqref{eq:gammaC},\eqref{eq:deltaC},\eqref{eq:yC}.
%\end{align*}

\begin{mini*}
{}{\sum_{(P,Q)\in E}\dvar{P}{Q}\yvar{P}{Q}}
{\label{eq:Example1}}{\tag{H-SPP-S}}
\addConstraint{\sum_{\{Q:(P, Q)\in E\}}\yvar{P}{Q}-\sum_{\{Q:(Q, P)\in E\}}\yvar{Q}{P}}{=\left\{
\begin{array}{rl} 
1, & \text{if } P\in\VS, \\
0, & \text{if } P\in\VB, \\
-1, & \text{if }P\in\VT.
\end{array}
\right.}
\addConstraint{\eqref{eq:alphaC},\eqref{eq:betaC},\eqref{eq:deltaC},\eqref{eq:gammaC},\eqref{eq:varepsilonC},\eqref{eq:yC}}{ }
\addConstraint{\eqref{eq:dC}, \eqref{eq:nC}.}{ }
\end{mini*}

The objective function minimizes the length of the path followed by the drone in the edges of the induced graph. The first constraints are the flow conservation constraints, the second constraints represent the sets $\ES$ and $\ET$ and the third ones state that the point selected must be in their respective neighborhoods.

\pgfplotsset{compat=1.15}
\usetikzlibrary{arrows}
\definecolor{ududff}{rgb}{0.30196078431372547,0.30196078431372547,1}
\definecolor{qqffqq}{rgb}{0,1,0}
\definecolor{qqqqff}{rgb}{0,0,1}
\definecolor{ffqqqq}{rgb}{1,0,0}
\begin{figure}[h!]
\centering
\begin{tikzpicture}[line cap=round,line join=round,>=triangle 45,x=1cm,y=1cm, scale=0.65]
\begin{axis}[
x=0.1cm,y=0.1cm,
axis lines=middle,
xmin=-5,
xmax=105,
ymin=-5,
ymax=105,
xtick={-30,-20,...,160},
ytick={-30,-20,...,95},]
\clip(-30.537729395742623,-29.679586779798765) rectangle (175.76223117610863,116.68410916363341);
\draw [line width=1pt,color=ffqqqq] (20,80)-- (40,30);
\draw [line width=1pt,color=ffqqqq] (70,95)-- (40,70);
\draw [line width=1pt,color=ffqqqq] (95,60)-- (60,70);
\draw [line width=1pt,color=ffqqqq] (60,50)-- (90,10);
\draw [line width=1pt,color=ffqqqq] (10,70)-- (20,50);
\draw [rotate around={0:(20,10)},line width=1pt,color=qqqqff,fill=qqqqff,fill opacity=0.25] (20,10) ellipse (1cm and 1cm);
\draw [rotate around={0:(90,90)},line width=1pt,color=qqffqq,fill=qqffqq,fill opacity=0.25] (90,90) ellipse (0.5cm and 0.5cm);
\draw [->,line width=1pt] (27.07,17.07) -- (40,30);
\draw [->,line width=1pt] (40,30) -- (60,70);
\draw [->,line width=1pt] (60,70) -- (85.36,88.14);
\draw (23.901825857302796,25.150339300103585) node[anchor=north west] {$\mathbf{P_S}$};
\draw (76.40719360287545,93.82231539794059) node[anchor=north west] {$\mathbf{P_T}$};
\begin{scriptsize}
\draw [color=ffqqqq] (20,80) circle (2.5pt);
\draw [color=ffqqqq] (40,30) circle (2.5pt);
\draw [color=ffqqqq] (70,95) circle (2.5pt);
\draw [color=ffqqqq] (40,70) circle (2.5pt);
\draw [color=ffqqqq] (95,60) circle (2.5pt);
\draw [color=ffqqqq] (60,70) circle (2.5pt);
\draw [color=ffqqqq] (60,50) circle (2.5pt);
\draw [color=ffqqqq] (90,10) circle (2.5pt);
\draw [color=ffqqqq] (10,70) circle (2.5pt);
\draw [color=ffqqqq] (20,50) circle (2.5pt);
\draw [fill=ududff] (27.07,17.07) circle (2.5pt);
\draw [fill=qqffqq] (85.36,88.14) circle (2.5pt);
\end{scriptsize}
\end{axis}
\end{tikzpicture}
\label{fig:solution_spp}
\caption{Solution for the instance of the \SPPN}
\end{figure}

\subsubsection{Reformulating the \SPP}\label{section:reformulation}
The problem exposed before can be reformulated by taking into account the following result:
\begin{prop}
There exists a finite dominant set (called $N_S^*$) in \SPP that represents $N_S$.
\end{prop}
\begin{proof}
Note that the point selected in $N_S$ in the optimal solution is the one that gives the minimum distance to the point of the barrier selected in the optimal solution.
\end{proof}
Therefore, we can compute the set $N_S^*$ by solving the convex problem for each endpoint of the barrier:
$$N_S^*=\{P_S(P_B^i)=\argmin_{P\in N_S} \|P_B^i - P\|:P_B^i\in V_{\mathcal B}\}.$$
Hence, the point selected $P_S$ can be represented by these points as shown
\begin{align*}\tag{N$^*$-C}\label{eq:nCs}
P_S&=\sum_{P_B^i\in V_{\mathcal B}}\mu_S(P_B^i)P_S(P_B^i),\\
1&=\sum_{P_B^i\in V_{\mathcal B}}\mu_S(P_B^i),
\end{align*}
where $\mu_S(P_B^i)$ is a binary variable that attains one if $P_S(P_B^i)$ is selected to exit from $N_S$.
The biggest advantage of this approach is that the whole graph $G$ is fixed and the incident edges can be computed for each point $P_S(P_B^i)$, $P_B^i\in V_{\mathcal B}$ separately. Defining again the variable $y^*$ for the edges in $E$, the new formulation for the \SPP can be expressed as the following program:
\begin{mini*}
{}{\sum_{(P,Q)\in E}\dvar{P}{Q}\yvar{P}{Q}}
{\label{eq:Example2}}{\tag{H-SPP-S$^*$}}
\addConstraint{\sum_{\{Q:(P, Q)\in E\}}\yvar{P}{Q}-\sum_{\{Q:(Q, P)\in E\}}\yvar{Q}{P}}{=\left\{
\begin{array}{rl} 
1, & \text{if } P\in\VS, \\
0, & \text{if } P\in\VB, \\
-1, & \text{if }P\in\VT.
\end{array}
\right.}
\addConstraint{\eqref{eq:dC}, \eqref{eq:nCs}.}{ }
\end{mini*}



\subsection{A formulation for the \TSP}
In the \TSP, the graph induced by the endpoints of the barriers and the neighborhoods is composed by the coming sets:

\begin{itemize}
\item $\VN=\{P_N:N\in\mathcal N\}$: set of the points selected in the neighborhoods in $\mathcal N$.
\item $\VB=\{P^1_B, P^2_B:B=\overline{P^1_B P^2_B}\in \mathcal B\}$: set of vertices that form the barriers of the problem.
\item $\EN=\{(P_N, P^i_{B}):P^i_B\in V_\B\text{ and } \overline{P_NP^i_B}\cap B''=\emptyset,\forall B''\in\B,\:i=1,2\}$: set of edges formed by the line segments that join the point selected in the neighborhood $N$ and every endpoint in the barriers and do not cross any barrier in $\B$.
\item $\EB=\{(P^{i}_B, P^{j}_{B'}):P^i_B, P^j_{B'}\in \VB \text{ and } \overline{P^i_B P^j_{B'}}\cap B''=\emptyset,\:\forall B''\in\mathcal B,\:i, j=1,2\}$: set of edges formed by the line segments that join two vertices of $V_{\mathcal B}$ and do not cross any barrier in $\B$.
\end{itemize} 

Following the same idea than before, we set $G=(V,E)$ induced by the barriers and the neighborhoords, where $V=\VN\cup\VB$ and $E=\EN\cup\EB$. 

The idea of the formulation of the \TSP is to considerate the variant called Steiner TSP (STSP) (references to this problem), where nodes in $\VB$ do not have to be visited, but they can be visited more than once if desired, and edges may be traversed more that once if desired. 

Note that, it is possible to convert any instance of the STSP into an instance of the standard TSP, by computing shortest paths between every pair of required nodes, when these nodes are fixed. In our problem, since the points in the neighborhoods are not fixed, this idea can not be applied to obtain the optimal solution for our problem, but it can produce a good approach to generate a good lower bound for the \TSP.


\newcommand{\gvar}[2]{g(#1#2)}

We can assume wlog that the neighborhood $N_1$ is required and the drone departs from the depot (that is assumed to be $N_1$) with $|\mathcal N|-1$ units of commodity that must be delivered to each required neighborhood. Then, for each edge $(P, Q)\in E$, we define the following variables:
\begin{itemize}
	\item $\yvar{P}{Q}$, binary variable that is equals to one if the drone goes from $P$ to $Q$.
	\item $\gvar{P}{Q}$, non-negative continuous variable that represents the amount of the commodity passing through the edge $(P, Q)$.
\end{itemize}

Hence, we can show the single-commodity flow formulation to the induced graph $G$ as follows:

\begin{mini*}
{}{\sum_{(P,Q)\in E}\dvar{P}{Q}\yvar{P}{Q}}
{\label{eq:Example1}}{\tag{H-TSP-S}}
\addConstraint{\sum_{\{Q:(P_N, Q)\in \EN\}}\yvar{P_N}{Q}}{\geq 1,}{\quad\forall P_N\in \VN}
\addConstraint{\sum_{\{Q:(P, Q)\in E\}}\yvar{P}{Q}}{= \sum_{\{Q:(Q, P)\in E\}}\yvar{Q}{P},}{\quad\forall P\in V}
\addConstraint{\sum_{\{Q:(Q, P_N)\in \EN\}}\gvar{Q}{P_N}-\sum_{\{Q:(P_N, Q)\in \EN\}}\gvar{P_N}{Q}}{= 1,}{\quad\forall P_N\in \VN\setminus\{P_{N_1}\}}
\addConstraint{\sum_{\{Q:(Q, P)\in E\}}\gvar{Q}{P}-\sum_{\{Q:(P, Q)\in E\}}\gvar{P}{Q}}{= 0,}{\quad\forall P\in \VB}
\addConstraint{\gvar{P}{Q}}{\leq (|\mathcal N|-1)\yvar{P}{Q},}{\quad\forall (P,Q)\in E}
\addConstraint{\eqref{eq:alphaC},\eqref{eq:betaC},\eqref{eq:deltaC},\eqref{eq:gammaC},\eqref{eq:varepsilonC},\eqref{eq:yC}}{ }
\addConstraint{\eqref{eq:dC}, \eqref{eq:nC}.}{ }
\end{mini*}

The first constraints impose that the drone departs from each neighborhood. The second constraints are the flow conservation constraints. The third inequalities ensure that one unit of commodity is delivered to each neighborhood where the the fourth ones conservate the amount of commodity for the points in the barriers. Finally, the last inequalities indicate that if any of the commodity passes along an edge, this edge must appears in the tour.

\pgfplotsset{compat=1.15}
\usetikzlibrary{arrows}
\definecolor{bfffqq}{rgb}{0.7490196078431373,1,0}
\definecolor{ffxfqq}{rgb}{1,0.4980392156862745,0}
\definecolor{qqffqq}{rgb}{0,1,0}
\definecolor{qqqqff}{rgb}{0,0,1}
\definecolor{ffqqqq}{rgb}{1,0,0}
\begin{figure}[h!]
\centering
\begin{tikzpicture}[line cap=round,line join=round,>=triangle 45,x=1cm,y=1cm, scale=0.65]
\begin{axis}[
x=0.1cm,y=0.1cm,
axis lines=middle,
xmin=-5,
xmax=105,
ymin=-5,
ymax=105,
xtick={-30,-20,...,160},
ytick={-30,-20,...,95},]
\clip(-30.537729395742623,-29.679586779798765) rectangle (175.76223117610863,116.68410916363341);
\draw [line width=1pt,color=ffqqqq] (20,80)-- (40,30);
\draw [line width=1pt,color=ffqqqq] (70,95)-- (40,70);
\draw [line width=1pt,color=ffqqqq] (95,60)-- (60,70);
\draw [line width=1pt,color=ffqqqq] (60,50)-- (90,10);
\draw [line width=1pt,color=ffqqqq] (10,70)-- (20,50);
\draw [rotate around={0:(20,10)},line width=1pt,color=qqqqff,fill=qqqqff,fill opacity=0.25] (20,10) ellipse (1cm and 1cm);
\draw [rotate around={0:(90,90)},line width=1pt,color=qqffqq,fill=qqffqq,fill opacity=0.25] (90,90) ellipse (0.5cm and 0.5cm);
\draw [rotate around={0:(35,85)},line width=1pt,color=ffxfqq,fill=ffxfqq,fill opacity=0.25] (35,85) ellipse (0.9cm and 0.9cm);
\draw [rotate around={0:(85,40)},line width=1pt,color=bfffqq,fill=bfffqq,fill opacity=0.25] (85,40) ellipse (1.1cm and 1.1cm);
\draw [->,line width=1pt] (27.07,17.07) -- (40,30);
\draw [->,line width=1pt] (40,30) -- (27.07,17.07);
\draw [->,line width=1pt] (40,30) -- (40,70);
\draw [->,line width=1pt] (40,70) -- (42.19,79.59);
\draw [->,line width=1pt] (42.19,79.59) -- (70,95);
\draw [->,line width=1pt] (70,95) -- (86.2,86.75);
\draw [->,line width=1pt] (86.2,86.75) -- (95,60);
\draw [->,line width=1pt] (95,60) -- (81.55,50.45);
\draw [->,line width=1pt] (81.55,50.45) -- (60,50);
\draw [->,line width=1pt] (60,50) -- (40,30);
\draw (20.94789460207184,18.16350179860903) node[anchor=north west] {$\mathbf{P_1}$};
\draw (86.35454360885365,93.50433831286361) node[anchor=north west] {$\mathbf{P_2}$};
\draw (36.349175863281,86.08483726584145) node[anchor=north west] {$\mathbf{P_3}$};
\draw (81.55301964690848,50.8774601568516) node[anchor=north west] {$\mathbf{P_4}$};
\begin{scriptsize}
\draw [color=ffqqqq] (20,80) circle (2.5pt);
\draw [color=ffqqqq] (40,30) circle (2.5pt);
\draw [color=ffqqqq] (70,95) circle (2.5pt);
\draw [color=ffqqqq] (40,70) circle (2.5pt);
\draw [color=ffqqqq] (95,60) circle (2.5pt);
\draw [color=ffqqqq] (60,70) circle (2.5pt);
\draw [color=ffqqqq] (60,50) circle (2.5pt);
\draw [color=ffqqqq] (90,10) circle (2.5pt);
\draw [color=ffqqqq] (10,70) circle (2.5pt);
\draw [color=ffqqqq] (20,50) circle (2.5pt);
\draw [fill=qqqqff] (27.07,17.07) circle (2.5pt);
\draw [fill=qqffqq] (86.2,86.75) circle (2.5pt);
\draw [fill=ffxfqq] (42.19,79.59) circle (2.5pt);
\draw [fill=bfffqq] (81.55,50.45) circle (2.5pt);
\end{scriptsize}
\end{axis}
\end{tikzpicture}
\label{fig:solution_tspp}
\caption{Solution for the instance of the \TSPN}
\end{figure}
%\textcolor{red}{¿Solucion del problema del TSP propuesto en el dibujo? }

%Note that this problem is not convex because the determinants that model the orientation of the points to check if there exist crossings produce product of variables. 

%\begin{equation*}
%\normalfont{\text{sign}}\left(\determinant{P_S^{}}{P_{B'}^1}{P_{B'}^2}\right) = \normalfont{\text{sign}}\left(\determinant{P_B^i}{P_{B'}^1}{P_{B'}^2}\right)
%\quad
%\text{or}
%\quad
%\normalfont{\text{sign}}\left(\determinant{P^{}_S}{P_{B'}^1}{P_{B'}^2}\right) = \normalfont{\text{sign}}\left(\determinant{P_B^i}{P_{B'}^1}{P_{B'}^2}\right),
%\quad\forall B'\in\B.
%\end{equation*}
%%$$=\emptyset,\quad\forall B''\in\B,$$
%\newcommand{\LS}[3]{L(#1|#2#3)}
%\newcommand{\US}[3]{U(#1|#2#3)}
%\newcommand{\alpha}[3]{\alpha^{+}(#1|#2#3)}
%\newcommand{\alphamenos}[3]{\alpha^{-}(#1|#2#3)}
%\newcommand{\alphacero}[3]{\alpha^{0\,}(#1|#2#3)}
%\textcolor{red}{Consideramos el caso en el que los puntos estan alineados?}
%
%To model the sign of the determinant $\determinant{P_S^{}}{P_{B'}^1}{P_{B'}^2}$, we define the following binary variables:
%\begin{itemize}
%\item $\alpha{P_S^{}}{P_{B'}^1}{P_{B'}^2}$, that is one if $\determinant{P_S^{}}{P_{B'}^1}{P_{B'}^2}$ is strictly positive and zero, otherwise.
%\item $\alphamenos{P_S^{}}{P_{B'}^1}{P_{B'}^2}$, that is one if $\determinant{P_S^{}}{P_{B'}^1}{P_{B'}^2}$ is strictly negative and zero, otherwise.
%\item $\alphacero{P_S^{}}{P_{B'}^1}{P_{B'}^2}$, that is one if $\determinant{P_S^{}}{P_{B'}^1}{P_{B'}^2}$ is zero and zero, otherwise.
%\end{itemize}
%
%
%It is possible to represent the sign condition by including the following constraints:
%%\begin{equation}\tag{sign+}\label{eq:sign}
%%\LS{P_S^{}}{P_{B'}^1}{P_{B'}^2}\left(1-\alpha{P_S^{}}{P_{B'}^1}{P_{B'}^2}\right)\leq \determinant{P_S^{}}{P_{B'}^1}{P_{B'}^2}\leq \US{P_S^{}}{P_{B'}^1}{P_{B'}^2}\alpha{P_S^{}}{P_{B'}^1}{P_{B'}^2},
%%\end{equation}
%\begin{align*}\tag{sign}\label{eq:sign}
%\determinant{P_S^{}}{P_{B'}^1}{P_{B'}^2}&\leq \US{P_S^{}}{P_{B'}^1}{P_{B'}^2}\:\alpha{P_S^{}}{P_{B'}^1}{P_{B'}^2},\\
%\determinant{P_S^{}}{P_{B'}^1}{P_{B'}^2}&\geq \LS{P_S^{}}{P_{B'}^1}{P_{B'}^2}\:\alphamenos{P_S^{}}{P_{B'}^1}{P_{B'}^2},\\
%\alpha{P_S^{}}{P_{B'}^1}{P_{B'}^2}+\alphamenos{P_S^{}}{P_{B'}^1}{P_{B'}^2}+\alphacero{P_S^{}}{P_{B'}^1}{P_{B'}^2}& = 1,
%\end{align*}
%where $\LS{P_S^{}}{P_{B'}^1}{P_{B'}^2}$ and $\US{P_S^{}}{P_{B'}^1}{P_{B'}^2}$ are a lower and a upper bound for the determinant, respectively. If the determinant is strictly positive, then $\alpha{P_S^{}}{P_{B'}^1}{P_{B'}^2}$ must be one to make the first inequality feasible, the second inequality is always fulfilled and the third one sets the other indicator variables to zero. The analagous case happens if the determinant is strictly negative. However, if the determinant is equal to zero, both inequalities are fulfilled and it is necessary to include the term $\epsilon (1-\alphacero{P_S^{}}{P_{B'}^1}{P_{B'}^2})$ in the objective function, where $\epsilon>0$ is a small value, to ensure that $\alphacero{P_S^{}}{P_{B'}^1}{P_{B'}^2}$ is equal to one when the determinant is zero.
%
%\newcommand{\betamas}[4]{\beta^{+}(#1#2|#3#4)}
%\newcommand{\betamenos}[4]{\beta^{-}(#1#2|#3#4)}
%\newcommand{\betacero}[4]{\beta^{0\,}(#1#2|#3#4)}
%
%At this point, it is required to introduce the binary variables that checks if both determinants are positive, negative or null:
%
%\begin{itemize}
%\item $\betamas{P_S^{}}{P_B^i}{P_{B'}^1}{P_{B'}^2}$, that is one if $\determinant{P_S^{}}{P_{B'}^1}{P_{B'}^2}$ and $\determinant{P_B^i}{P_{B'}^1}{P_{B'}^2}$ are both positive.
%\item $\betamenos{P_S^{}}{P_B^i}{P_{B'}^1}{P_{B'}^2}$, that is one if $\determinant{P_S^{}}{P_{B'}^1}{P_{B'}^2}$ and $\determinant{P_B^i}{P_{B'}^1}{P_{B'}^2}$ are both negative.
%\item $\betacero{P_S^{}}{P_B^i}{P_{B'}^1}{P_{B'}^2}$, that is one if $\determinant{P_S^{}}{P_{B'}^1}{P_{B'}^2}$ and $\determinant{P_B^i}{P_{B'}^1}{P_{B'}^2}$ are both zero.
%\end{itemize}
%
%Hence, the $\beta$ variables can be represented by the product of the $\alpha$ variables:
%\begin{align*}
%\betamas{P_S^{}}{P_B^i}{P_{B'}^1}{P_{B'}^2}&=\alpha{P_{B'}^1}{P^{}_S}{P^i_B}\:\alpha{P_{B'}^2}{P^{}_S}{P^i_B},\\
%\betamenos{P_S^{}}{P_B^i}{P_{B'}^1}{P_{B'}^2}&=\alphamenos{P_{B'}^1}{P^{}_S}{P^i_B}\:\alphamenos{P_{B'}^2}{P^{}_S}{P^i_B},\\
%\betacero{P_S^{}}{P_B^i}{P_{B'}^1}{P_{B'}^2}&=\alphacero{P_{B'}^1}{P^{}_S}{P^i_B}\:\alphacero{P_{B'}^2}{P^{}_S}{P^i_B}.
%\end{align*}
%
%These products can be linearized by means of these constraints:
%$$
%\left\{\begin{array}{rcl}
%\betamas{P_S^{}}{P_B^i}{P_{B'}^1}{P_{B'}^2} & \leq & \alpha{P_{B'}^1}{P^{}_S}{P^i_B},\\\\
%\betamas{P_S^{}}{P_B^i}{P_{B'}^1}{P_{B'}^2} & \leq & \alpha{P_{B'}^2}{P^{}_S}{P^i_B},\\\\
%\betamas{P_S^{}}{P_B^i}{P_{B'}^1}{P_{B'}^2} & \geq & \alpha{P_{B'}^1}{P^{}_S}{P^i_B} + \alpha{P_{B'}^2}{P^{}_S}{P^i_B} - 1.\end{array}\right.
%$$
%
% $\alpha{P_S^{}}{P_{B'}^1}{P_{B'}^2}$ that attains minus one if the sign of the determinant is negative, zero, if the determinant is zero, and one, if the determinant is positive. The following constraints set the sign condition:
%\begin{equation}\tag{sign}\label{eq:sign}
%\LS{P_S^{}}{P_{B'}^1}{P_{B'}^2}\:\alpha{P_S^{}}{P_{B'}^1}{P_{B'}^2}\leq \determinant{P_S^{}}{P_{B'}^1}{P_{B'}^2}\leq \US{P_S^{}}{P_{B'}^1}{P_{B'}^2})\left(1-\alpha{P_S^{}}{P_{B'}^1}{P_{B'}^2}\right),
%\end{equation}
% If the determinant is negative, then the right side of the equation must be zero, and $\alpha{P_S^{}}{P_{B'}^1}{P_{B'}^2}$ is one. Analogously, if the determinant is non-negative, then the left side must be zero and $\alpha{P_S^{}}{P_{B'}^1}{P_{B'}^2}$ is also zero.
%
%%In the left picture, the number of incident edges $|E_S|$ on $N_S$ is 4 whereas the right picture is 6.

\section{Strengthening the formulations}
\textcolor{red}{Incluir el resultado estructural que da una cota superior del numero de bolas que se puede generar}
\subsection{Preprocessing}
In this subsection, we show a preprocessing result that allows to fix some variables by taking into account the relative position between the neighborhoods and the barriers.
In particular, we are going to present an outcome that ensures that there are some barriers whose endpoints can not be incident in the edges of $\EN$ and it is not necessary to include it in $\EN$. 

Let denote

\begin{align*}
\text{cone}(P,Q,R)&:=\{\mu_1 \overrightarrow{PQ}+\mu_2\overrightarrow{PR}:\mu_1,\mu_2\geq 0\},\\
\text{cone}(P,Q,R)^-&:=\{\mu_1 \overrightarrow{PQ}+\mu_2\overrightarrow{PR}:\mu_1,\mu_2\geq 0,\:\mu_1+\mu_2\leq 1\},\\
\text{cone}(P,Q,R)^+&:=\{\mu_1 \overrightarrow{PQ}+\mu_2\overrightarrow{PR}:\mu_1,\mu_2\geq 0,\:\mu_1+\mu_2\geq 1\}.
\end{align*}
Note that, $cone(P, Q, R)$ is the union of $\text{cone}(P,Q,R)^-$ and $\text{cone}(P,Q,R)^+$. It is also important to remark that $\text{cone}(P,Q,R)^+$ is the part of cone that crosses the barrier $\overline{QR}$ when we consider a segment whose endpoints are $P$ and another point of this set, i.e.
$$\text{cone}(P,Q,R)^+=\{P':\overline{PP'}\cap\overline{QR}\neq\emptyset\}.$$

Let $B = \overline{P^1_BP^2_B}\in\mathcal B$ a barrier. In the following proposition, we give a sufficient condition to not include the edge $(P^{}_N, P^i_B)$ in $\EN$:

\begin{prop}

Let $B'=\overline{P^1_{B'}P^2_{B'}}\in\mathcal B$ and $\normalfont{\text{cone}}(P^i_B, P^1_{B'}, P^2_{B'})^+$ the conical hull generated by these points. If
$$N\subset\bigcup_{B'\in\mathcal B}\normalfont{\text{cone}}(P^i_B, P^1_{B'}, P^2_{B'})^+,$$
then $(P^{}_N, P^i_B)\not\in \EN$.

%Let $B'=\overline{P^1_{B'}P^2_{B'}}\in\mathcal B$ and $Q(B')$ the intersection point of the straight lines that join points of $N$ and $B'$ that lies outside of the convex hull generated by thse line segments. Let $\normalfont{\text{cone}}(\{\overrightarrow{Q(B')P^1_N},\overrightarrow{Q(B')P^2_N}\})$ the conical hull of these vectors. If
%$$B\subset\bigcup_{B'\in\mathcal B}\normalfont{\text{cone}}(\{\overrightarrow{Q(B')P^1_N},\overrightarrow{Q(B')P^2_N}\}),$$
%then $(P^{}_N, P^i_{B})\not\in \EN$, $i=1,2$.
\end{prop}
\begin{proof}
If $P_N\in N$, then there exists a $B'\in\mathcal B$ such that 
$P_N\in \normalfont{\text{cone}}(P^i_B, P^1_{B'}, P^2_{B'})^+$. Therefore, $\overline{P^i_B P^{}_N}\cap B'\neq\emptyset$ and $(P^{}_N, P^i_B)\not\in \EN$, as we claimed.
\end{proof}

We can check computationally the condition of the previous proposition by using the following procedure. The first issue that is solved is the one when the neighborhoods are segments. Let $N = \overline{P^1_N P^2_N}$ be a line segment and $r_N$ the straight line that contains the line segment $N$ that is represented as:

$$r_N:P_N^1+\lambda\overrightarrow{P_N^1P_N^2},\qquad\lambda\in\mathbb R.$$


\begin{algorithm}[H]
\caption{Checking computationally if $(P^{}_N, P^i_B)\not\in \EN$ when $N$ is a segment.}
\label{alg:Algorithm1}
\SetKwInOut{Input}{Initialization}\SetKwInOut{Output}{output}

 \Input{Let $P^i_B$ be the point whose edge $(P^i_B, P^{}_N)$ is going to check if $(P^{}_N, P^i_B)\not\in \EN$.\\
Set $points = \{P^1_N, P^2_N\}$, $lambdas=\{0, 1\}$.}
% Set $LB=0$, $UB=+\infty$, $\bar z=z^0$.
\For{$B''\in\mathcal B$}{
	\For{$j\in\{1, 2\}$}{
		Compute the straight line 
		$$r(P^i_B, P^j_{B''}) = P^i_B + \mu^j_{B''}\overrightarrow{P^i_BP^j_{B''}},$$ 
		that contains the points $P^i_B$ and $P^j_{B''}$. \\
	
		Intersect $r(P^i_B, P^j_{B''})$ and $r_N$ in the point $Q^j_{B''}$ and compute $			\overline{\mu}^j_{B''}$ such that 
		$$Q^j_{B''}=P_B^i + \overline{\mu}^j_{B''} \overrightarrow{P^i_BP^j_{B''}}.$$\\
	
		\If{$|\overline{\mu}^j_{B''}|\geq 1$}{
			Compute $\lambda^j_{B''}$ such that 
			$$Q^j_{B''}=P^1_N+\lambda^j_{B''}\overrightarrow{P^1_NP^2_N}.$$\\
			\If{$\overline{\mu}^j_{B''}\geq 1$}{
				Include $\lambda^j_{B''}$ in $lambdas$.}
			\Else{
				\If{$\lambda^j_{B''}\geq 0$}{
					Set $\lambda^j_{B''}=M<<0$ and include it in $lambdas$.}
				\Else{
					Set $\lambda^j_{B''}=M>>0$ and include it in $lambdas$.}
	   			}	
			}
		}
	}
Order in non-decreasing order the set $lambdas$.\\
If it is verified that
\begin{align*}
\min\{\lambda_{B'}^1, \lambda_{B'}^2\}\leq 0\leq\max\{\lambda_{B'}^1, \lambda_{B'}^2\},&\quad\text{ for some } B'\in\mathcal B,\\
\min\{\lambda_{B'}^1, \lambda_{B'}^2\}\leq 1\leq\max\{\lambda_{B'}^1, \lambda_{B'}^2\},&\quad\text{ for some } B'\in\mathcal B,\\
\min\{\lambda_{B'}^1, \lambda_{B'}^2\}\leq\lambda_{B''}^j\leq\max\{\lambda_{B'}^1, \lambda_{B'}^2\},&\quad\text{ for some } B'\in\mathcal B\setminus\{B''\},\quad\forall \lambda_{B''}^j\in lambdas\setminus\{M\},
\end{align*}
or
$$
\min\{\lambda_{B'}^1, \lambda_{B'}^2\}\leq 0, 1\leq\max\{\lambda_{B'}^1, \lambda_{B'}^2\},\quad\text{ for some } B'\in\mathcal B,
$$
then $(P^{}_N, P^i_B)\not\in \EN$.

\end{algorithm}

\input{figures/Preprocessing - First case}

Note that this algorithm also allows us to decide if the drone can access to a point of a barrier from any point of the neighborhood $N$. It is enough to check in (15) that 
$$0\not\in \left[\min\{\lambda_{B'}^1, \lambda_{B'}^2\},\max\{\lambda_{B'}^1, \lambda_{B'}^2\}\right]\quad\text{and}\quad 1\not\in \left[\min\{\lambda_{B'}^1, \lambda_{B'}^2\},\max\{\lambda_{B'}^1, \lambda_{B'}^2\}\right],\quad\forall B'\in\mathcal B.$$

For the case when $N$ is an ellipse, the same rationale can be followed. The idea is to generate the largest line segment contained in the ellipse and repeat the procedure exposed in the Algorithm \ref{alg:Algorithm1}. Let $F_1$ and $F_2$ the focal points of $N$.

\begin{algorithm}[H]
\caption{Checking computationally if $(P^{}_N, P^i_B)\not\in \EN$ when $N$ is an ellipse.}
\label{alg:Algorithm2}
\SetKwInOut{Input}{Initialization}\SetKwInOut{Output}{output}

 \Input{Let $P^i_B$ be the point whose edge $(P^i_B, P^{}_N)$ is going to check if $(P^{}_N, P^i_B)\not\in \EN$.\\
Set $points = \{\}$, $lambdas=\{\}$.}
% Set $LB=0$, $UB=+\infty$, $\bar z=z^0$.}
Compute the straight line $r(F^1, F^2)$.\\
Intersect $r(F^1, F^2)$ and $\partial N$ in the points $P_N^1$ and $P_N^2$. \\
Include $P_N^1$ and $P_N^2$ in $points$. \\
Apply the Algorithm \ref{alg:Algorithm1}.
\end{algorithm}


\textcolor{red}{Estudiar resultados que eliminen algunas de las posibles aristas de $E_S$ y $E_T$}
\textcolor{red}{Hablar que las aristas $E_{\mathcal B}$ se pueden preprocesar porque los puntos están fijados}

\subsection{Valid inequalities}
This subsection is devoted to show some results that adjust the big-M constants that appear in the previous formulation, specifically, in the \eqref{eq:alphaC}, where the modelling of the sign requires to compute the lower and upper bounds $L$ and $U$, respectively. We are going to determine these bounds explicitly for the cases when the neighborhoods are ellipses and segments.

Let $\overline{P^1_{B'}P^2_{B'}}=B'\in\B$ be a barrier and $P_N\in N$. Let $\determinant{P_N}{P_{B'}^1}{P_{B'}^2}$ also be the determinant whose values must be bounded. Hence, the solution of the following problem gives the lower bound of the determinant:
\begin{equation*}\label{eq:L-Problem}\tag{L-Problem}
\overline{L}=\min_{P_N=(x,y)\in N}F(x,y):=\determinant{P^{}_N}{P_{B'}^1}{P_{B'}^2}=\left|
\begin{array}{cc}
P^{1}_{B'_x}-x & P^{2}_{B'_x}-x \\
P^{1}_{B'_y}-y & P^{2}_{B'_y}-y
\end{array}
\right|.
\end{equation*}

\subsubsection{Lower and upper bounds when the neighborhoods are line segments}
In this case, the segment whose endpoints are $P^1_{N_S}$ and $P^2_{N_S}$ can be expressed as the following convex set:
$$N=\{(x,y)\in\mathbb R^2:(x,y)=\mu P^1_{N}+(1-\mu)P^2_{N}, 0\leq\mu\leq1\}.$$
Since we are optimizing a linear function in a compact set we can conclude that the objective function in \eqref{eq:L-Problem} achieves its minimum and its maximum in the extreme points of $N_S$, that is, in $P^1_{N}$ and $P^2_{N}$. 

\subsubsection{Lower and upper bounds when the neighborhoods are ellipsoids}
The first case that is considered is the one when $N$ is an ellipse, that is, $N$ is represented by the following inequality:
$$N=\{(x,y)\in\mathbb R^2:ax^2+by^2+cxy+dx+ey+f\leq 0\}=$$
where $a, b, c, d, e, f$ are coefficients of the ellipse.
In an extended form, we need to find:
\begin{mini*}
{}{F(x, y)=\left|
\begin{array}{cc}
P^{1}_{B'_x}-x & P^{2}_{B'_x}-x \\
P^{1}_{B'_y}-y & P^{2}_{B'_y}-y
\end{array}
\right|=xP^{1}_{B'_y}-xP^{2}_{B'_y}+yP^{2}_{B'_x}-yP^{1}_{B'_x}+P^{1}_{B'_x}P^{2}_{B'_y}-P^{1}_{B'_y}P^{2}_{B'_x},}
{\label{eq:Example1}}{\tag{L-Ellipse}}
\addConstraint{ax^2+by^2+cxy+dx+ey+f\leq 0.}
\end{mini*}
Since we are minimizing a linear function in a convex set, we can conclude that the extreme points are located in the frontier, so we can use the Lagrangian function to compute these points.
$$F(x,y;\lambda)=xP^{1}_{B'_y}-xP^{2}_{B'_y}+yP^{2}_{B'_x}-yP^{1}_{B'_x}+P^{1}_{B'_x}P^{2}_{B'_y}-P^{1}_{B'_y}P^{2}_{B'_x}+\lambda(ax^2+by^2+cxy+dx+ey+f).$$

$$\nabla F(x,y;\lambda)=0\Longleftrightarrow
\left\{\begin{array}{rcll}
\frac{\partial F}{\partial x} & = & P^{1}_{B'_y}-P^{2}_{B'_y}+2ax\lambda+cy\lambda+d\lambda& =0,\\
\frac{\partial F}{\partial y} & = & P^{2}_{B'_x}-P^{1}_{B'_x}+2by\lambda+cx\lambda+e\lambda& =0,\\
\frac{\partial F}{\partial \lambda} & = & ax^2+by^2+cxy+dx+ey+f& =0.\\
\end{array}\right.$$
From the first two equations we can obtain:

$$\lambda = \frac{P^{2}_{B'_y}-P^{1}_{B'_y}}{2ax+cy+d}=\frac{P^{1}_{B'_x}-P^{2}_{B'_x}}{2by+cx+e}.$$
From this equality, we can express $y$ as a function of $x$:
\begin{align*}
(P^{2}_{B'_y}-P^{1}_{B'_y})(2by+cx+e)&=(P^{1}_{B'_x}-P^{2}_{B'_x})(2ax+cy+d),\\
y\left[2b(P^{2}_{B'_y}-P^{1}_{B'_y})-c(P^{1}_{B'_x}-P^{2}_{B'_x})\right]&=\left[2a(P^{1}_{B'_x}-P^{2}_{B'_x})-c(P^{2}_{B'_y}-P^{1}_{B'_y})\right]x+\left[d(P^{1}_{B'_x}-P^{2}_{B'_x})-e(P^{2}_{B'_y}-P^{1}_{B'_y})\right],\\
y&=\left[\frac{2a(P^{1}_{B'_x}-P^{2}_{B'_x})-c(P^{2}_{B'_y}-P^{1}_{B'_y})}{2b(P^{2}_{B'_y}-P^{1}_{B'_y})-c(P^{1}_{B'_x}-P^{2}_{B'_x})}\right]x+\left[\frac{d(P^{1}_{B'_x}-P^{2}_{B'_x})-e(P^{2}_{B'_y}-P^{1}_{B'_y})}{2b(P^{2}_{B'_y}-P^{1}_{B'_y})-c(P^{1}_{B'_x}-P^{2}_{B'_x})}\right],\\
y&=mx+n.
\end{align*}
Finally, we can substitute $y$ in the third equation to compute the value of $x$:
\begin{align*}
ax^2+by^2+cxy+dx+ey+f &= ax^2+b(mx+n)^2+cx(mx+n)+dx+e(mx+n)+f =\\
&= (a+bm^2+cm)x^2+(2bmn+cn+d+em)x+(n^2b+en+f)=0.
\end{align*}
By using the standard form of the solution of a quadratic equation:
\begin{align*}
x^{\pm}&=\frac{-(2bmn+cn+d+em)\pm\sqrt{(2bmn+cn+d+em)^2-4(a+bm^2+cm)(n^2b+en+f)}}{2(a+bm^2+cm)},\\
y^{\pm}&=mx^{\pm}+n.
\end{align*}
Hence, to compute the lower and upper bound, we only need to evaluate $x^{\pm}$ and $y^{\pm}$ in the objective function and the lowest and highest value correspond to $\LS{P_N^{}}{P_{B'}^1}{P_{B'}^2}$ and $\US{P_N^{}}{P_{B'}^1}{P_{B'}^2}$, respectively.

\end{document}